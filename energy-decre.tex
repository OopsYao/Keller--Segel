\documentclass{article}

\usepackage{fontspec}
\usepackage{amsmath}
\usepackage{unicode-math}
\usepackage{geometry}
\usepackage{hyperref}
\usepackage{cleveref}

% Font settings
\setmainfont{texgyrepagella}[
  Extension = .otf,
  UprightFont = *-regular,
  BoldFont = *-bold,
  ItalicFont = *-italic,
  BoldItalicFont = *-bolditalic,
]
\setsansfont{texgyreheros}[
  Extension = .otf,
  UprightFont = *-regular,
  BoldFont = *-bold,
  ItalicFont = *-italic,
  BoldItalicFont = *-bolditalic,
]
\setmonofont{inconsolata}
\setmathfont{texgyrepagella-math.otf}

% Geometry
\geometry{
  a4paper,
  left=3cm,
  top=3cm,
  right=3cm,
  bottom=3cm,
}

% Remove page number of the 1st page
\let\oldmaketitle\maketitle
\renewcommand{\maketitle}
  {\oldmaketitle\thispagestyle{empty}}

\newcommand{\diff}{\mathrm d}


\title{Keller--Segel Notes 2}

\def\diff{\mathrm d}
\newtheorem{lemma}{Lemma}

\begin{document}
Our system is given by
\begin{equation}
  \label{eq:system}
  \left\{
  \begin{aligned}
    u_t & = (u(u - \chi v)')', x \in (0, L), t > 0 \\
    v_t & = v'' - v + u, x \in (0, L), t > 0       \\
  \end{aligned}
  \right.
\end{equation}
with homegenous NBC.
And the first equation of \cref{eq:system} in the form of pseudo-inverse distribution
function is given by
\begin{equation}
  \partial_t\Phi = -\frac12\left(\frac1{\Phi'^2}\right)'+\chi v'(\Phi).
\end{equation}
It is worth noting that all the space variable derivatives are represented by the prime notation,
while the time variable derivatives are represented by subscript $t$ or $\partial_t$.

\section{Energy Decreasing}
The energy of the system is given by
\begin{equation}
  \label{eq:energy}
  \mathcal E(u, v) = \frac 1\chi \int_0^L u^2\diff x + \int_0^L(v'^2 + v^2 - 2uv)\diff x,
\end{equation}
and in the form of pseudo-inverse distribution function $\Phi$ it writes
\def\hatE{\hat{\mathcal E}}
\begin{equation}
  \hat{\mathcal E}(\Phi, v) = \mathcal E(u, v) =
  \frac 1\chi \int_0^{M_u} \frac 1{\Phi'}\diff\eta
  - 2\int_0^{M_u} v(\Phi)\diff\eta
  + \int_0^L (v'^2 + v^2)\diff x,
\end{equation}
where $M_u$ denotes the total mass of $u$.

We want to find a splitting scheme such that the energy of the system is decreasing.
And we shall see later that this is a result mainly due to the convex structure of the
energy. To make it more clear, we focus on the semi-discrete case, i.e., the discretization
only happens in the time dimension, while the space is continuous. Then the full discrete case
in both time and space is just an analog of it.

The Strang-splitting introduces two operators $T$ and $S$ s.t. the scheme is given
by
\[
  (\Phi^{k + 1}, v^{k + 1}) = T_{\tau/2}S_{\tau}T_{\tau/2}(\Phi^k, v^k),
\]
here the superscript $k$ means the $k$-th step and $\tau$ for time step.
Furthermore, the operator $T$ only changes $\Phi$ and $S$ only changes $v$.
It follows that all we need is to prove that the energy decreases w.r.t.
variables $\Phi$ and $v$ after operation $T$ and $S$ respectively.
Before proceeding, we shall extend the convexity
inequality into general vector space.
\begin{lemma}\label{lemma:convex ineq}
  Let $\mathcal E: U \to\mathbb R$ be a functional on a convex subset $U$ of the vector space $V$
  (with the field $\mathbb R$), and for $f, g\in U$, if
  $\varphi(t) = \mathcal E(tf + (1-t)g), t\in[0, 1]$ (or equivalently, $\varphi(t) = \mathcal E(tg + (1-t)f)$)
  is a convex differentiable function, then we have that
  \[
    \frac{\delta\mathcal E}{\delta g}(f - g)
    \leq \mathcal E(f) - \mathcal E(g)
    \leq\frac{\delta\mathcal E}{\delta f}(f - g).
  \]
  Here the notation $\delta\mathcal E/\delta f$ denotes the Fr\'echet derivative of
  $\mathcal E$ at $f$.
\end{lemma}
\begin{proof}
  Consider the function $\varphi(t) = \mathcal E(f + t(g - f))$, and at the point $t = 0$ and $t = 1$,
  by the convexity we have that
  \[
    \varphi(1) - \varphi(0) \geq \varphi'(0),
  \]
  i.e., the right inequality,
  \[\mathcal E(g) - \mathcal E(f) \geq \frac{\delta\mathcal E}{\delta f}(g - f),\]
  and the symmetry of $f$ and $g$ yields the left one.
\end{proof}

\subsection{Semi-Discrete Case}
We shall see that the energy has a non-convexity structure w.r.t. $\Phi$ due to the term $v$:
\[
  \begin{aligned}
    \frac{\diff^2}{\diff t^2}\hat{\mathcal E}(t\Phi + (1-t)\Psi, v)
     & = \frac 2{\chi}\int_0^{M_u} \frac{(\Phi - \Psi)'^2}{(t\Phi' + (1-t)\Psi')^3}\diff\eta
    - 2\int_0^{M_u}v''(t\Phi + (1-t)\Psi)(\Phi-\Psi)^2\diff\eta,                             \\
    \frac{\diff^2}{\diff t^2}\mathcal E(u, tv + (1-t)w)
     & =2\int_0^L \left((v-w)'^2 + (v-w)^2\right)\diff x,
  \end{aligned}
\]
however, the convex-splitting technique enables us to match structure requirement.
To this end, we write the function $v = v_c - v_e$ as the difference of convex functions.
And split the energy $\hatE$ s.t. $\hatE = \hatE_1 - \hatE_2$, where
\[
  \begin{aligned}
    \hatE_1(\Phi, v) & = \frac 1\chi \int_0^{M_u} \frac 1{\Phi'}\diff\eta
    + 2\int_0^{M_u}v_e(\Phi)\diff\eta
    + \int_0^L(v'^2 + v^2)\diff x,                                        \\
    \hatE_2(\Phi, v) & = 2\int_0^{M_u}v_c(\Phi)\diff\eta.
  \end{aligned}
\]
Then it easy to verify that $\hatE_1$ and $\hatE_2$ are convex
as $\Phi',\Psi'\geq 0$.

Suppose that $T(\Phi, v) = (\tilde\Phi, v), S(\Phi, v) = (\Phi, \tilde v)$,
then by \cref{lemma:convex ineq}, we have that
\[
  \hatE_1(\tilde\Phi, v) - \hatE_1(\Phi, v)
  \leq\frac{\delta\hatE_1}{\delta\tilde\Phi}(\tilde\Phi - \Phi)
  = -\frac 1\chi \int_0^{M_u} \frac{(\tilde \Phi -\Phi)'}{\tilde\Phi'^2}\diff\eta
  + 2\int_0^{M_u}v_e'(\tilde\Phi)(\tilde \Phi-\Phi)\diff\eta.
\]
With that $\Phi, \tilde\Phi$ share the same DBC, integration by parts gives us
\[
  \hatE_1(\tilde\Phi, v) - \hatE_1(\Phi, v) \leq
  \frac 1\chi\int_0^{M_u}\left(\frac 1{\tilde\Phi'^2}\right)'(\tilde\Phi - \Phi)\diff\eta
  + 2\int_0^{M_u}v_e'(\tilde\Phi)(\tilde \Phi-\Phi)\diff\eta.
\]
We also have that
\[
  \hatE_2(\tilde\Phi, v) - \hatE_2(\Phi, v)
  \geq \frac{\delta\hatE_2}{\delta\Phi}(\tilde\Phi - \Phi)
  =2\int_0^{M_u}v_c'(\Phi)(\tilde\Phi - \Phi)\diff\eta.
\]
It follows that
\[
  \hatE(\tilde\Phi, v) - \hatE(\Phi, v)
  \leq -\frac 2\chi\int_0^{M_u}
  \left(-\frac 12\left(\frac 1{\tilde\Phi'^2}\right)'
  + \chi\left(v_c'(\tilde\Phi) - v_e'(\Phi)\right)\right)
  (\tilde\Phi - \Phi)\diff\eta.
\]
That being said, if the scheme is given by
\begin{equation}
  \label{eq:semi scheme of Phi}
  \frac{\tilde\Phi - \Phi}{\tau} = -\frac 12\left(\frac 1{\tilde\Phi'^2}\right)'
  + \chi\left(v_c'(\Phi) - v_e'(\tilde\Phi)\right),
\end{equation}
then we have the energy decreasing w.r.t. the variable $\Phi$,
\[
  \hatE(\tilde\Phi, v) - \hatE(\Phi, v)
  \leq -\frac 2{\chi\tau}\int_0^{M_u}(\tilde\Phi - \Phi)^2\diff\eta
  \leq 0.
\]

Similarly, the homogeneous NBC of $v$ and the scheme
\begin{equation}
  \label{eq:semi scheme of v}
  \frac{\tilde v - v}{\tau} = \tilde v'' - \tilde v + u
\end{equation}
will lead to the energy decreasing w.r.t. the variable $v$,
\[
  \mathcal E(u, \tilde v) - \mathcal E(u, v)
  \leq -\frac 2{\tau}\int_0^L (\tilde v - v)^2\diff x
  \leq 0.
\]

So we conclude that the scheme given by \cref{eq:semi scheme of Phi,eq:semi scheme of v}
holds the energy decreasing property.

\subsection{The Full Discretization}
This situation is an analog of the continuous case, and here we reuse the same notations
like $\mathcal E, \hatE$ etc.
The energy is given by
\[
  \begin{aligned}
    \hatE(\Phi, v)   & = \frac{h^2}\chi\sum_{i = 0}^{N - 1}\frac 1{\Phi_{i+1}-\Phi_i}
    - 2h\sum_{i = 0}^N v(\Phi_i)
    + h\sum_{i = 0}^{N-1} \left(\left(\frac{v_{i+1} - v_i}h\right)^2 + v_i^2\right),  \\
    \mathcal E(u, v) & = h\sum_{i = 0}^{N - 1}\left(
    \left(\frac{v_{i+1} - v_i}h\right)^2 + v_i^2 - 2u_iv_i
    \right),
  \end{aligned}
\]
and the scheme of $T$ is given by
\[
  \frac{\tilde\Phi_i - \Phi_i}{\tau} =
  -\frac{\left(\frac{h}{\tilde\Phi_{i+1} - \tilde\Phi_i}\right)^2
    -\left(\frac{h}{\tilde\Phi_{i} - \tilde\Phi_{i-1}}\right)^2}{2h}
  +\chi\left(v'_c(\Phi_i) - v'_e(\tilde\Phi_i)\right),
  \quad i = 1,\ldots, N-1,
\]
with the DBC $\tilde\Phi_0 = \Phi_0$ and $\tilde\Phi_N = \Phi_N$.
And the scheme of $S$ is given by
\[
  \frac{\tilde v_i -v_i}{\tau} =
  \frac{\tilde v_{i+1} - 2\tilde v_i + \tilde v_{i-1}}{h^2}
  - \tilde v_i + u_i,
  \quad i = 1,\ldots, N-1,
\]
with the homegeneous NBC $\tilde v_0 = \tilde v_1$ and $\tilde v_N = \tilde v_{N - 1}$.

By \cref{lemma:convex ineq}, together with the boundary conditions and summation by
parts technique, we can easily verify that the energy is decreasing,
\[\begin{aligned}
    \hatE(\tilde\Phi, v) - \hatE(\Phi, v)
    &\leq -\frac{2h}{\chi\tau}\sum_{i = 1}^{N-1}\left(\tilde\Phi_i - \Phi_i\right)^2
    \leq 0, \\
    \mathcal E(u, \tilde v) - \mathcal E(u, v)
    &\leq -\frac{2h}{\tau}\sum_{i = 1}^{N-1}\left(\tilde v_i - v_i\right)^2
    \leq 0.
  \end{aligned}\]
\end{document}
