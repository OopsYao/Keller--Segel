\documentclass{article}
\usepackage{fontspec}
\usepackage{amsmath}
\usepackage{amsthm}
\usepackage[bold-style=ISO]{unicode-math}
\usepackage{geometry}
\usepackage{hyperref}
\usepackage{cleveref}
\usepackage{booktabs}
\usepackage{siunitx}
\usepackage{biblatex}

% Font settings
\setmainfont{texgyrepagella}[
  Extension = .otf,
  UprightFont = *-regular,
  BoldFont = *-bold,
  ItalicFont = *-italic,
  BoldItalicFont = *-bolditalic,
]
\setsansfont{texgyreheros}[
  Extension = .otf,
  UprightFont = *-regular,
  BoldFont = *-bold,
  ItalicFont = *-italic,
  BoldItalicFont = *-bolditalic,
]
\setmonofont{inconsolata}
\setmathfont{texgyrepagella-math.otf}

% Geometry
\geometry{
  a4paper,
  left=3cm,
  top=3cm,
  right=3cm,
  bottom=3cm,
}

% Remove page number of the 1st page
\let\oldmaketitle\maketitle
\renewcommand{\maketitle}
  {\oldmaketitle\thispagestyle{empty}}

\newcommand{\diff}{\mathrm d}
\DeclareMathOperator{\sgn}{sgn}
\let\vec\symbf


\title{Keller--Segel Notes}
\author{xyao}

\newcommand{\bigo}{\mathcal O}
\newcommand{\frender}{\mathcal D}

\begin{document}
  \maketitle
  \section{Splitting method}
  Consider the problem
  \[
    \partial_t\phi = A\phi,\quad t\in(t_n, t_{n + 1})
  \]
  where $A$ is an operator, then the solution writes $\phi_{t_{n + 1}} = \exp\{(t_{n + 1} - t_n)A\}
  \phi_{t_n} = \exp\{hA\}\phi_{t_n}$.
  If $A$ has decomposition $A = A_1 + A_2$, then $\exp\{h(A_1 + A_2)\}$ and $\exp\{hA_1\}\exp\{hA_2\}$
  inspire us to decouple the system to two subproblems,
  \[
    \begin{aligned}
      \partial_t\phi^* &= A_1\phi^*,&\quad\phi^{*}(t_n) &= \phi(t_n), \\
      \partial_t\phi^{**} &= A_2\phi^{**},&\quad \phi^{**}(t_n) &= \phi^{*}(t_{n + 1}).
    \end{aligned}
  \]

  The Taylor's expansion of the operator writes
  \[
    \mathrm e^{hA} = I + hA + \frac{h^2}2 A^2 + \bigo(h^3).
  \]
  So for the operator $\exp\{h(A_1 + A_2)\}$,
  \[
    \mathrm e^{h(A_1 + A_2)} = I + h(A_1 + A_2) + \frac{h^2}2 (A_1^2 + A_2^2 + A_1A_2 + A_2A_1) + \bigo(h^3),
  \]
  Consider the splitting operators,
  \[
    \begin{aligned}
      \mathrm e^{hA_2}\mathrm e^{hA_1}
      &= I + h(A_1 + A_2) + \frac{h^2}2 (A_1^2 + A_2^2 + 2A_2A_1) + \bigo(h^3), \\
      \mathrm e^{\frac h2A_1}\mathrm e^{hA_2}\mathrm e^{\frac h2A_1}
      &= I + h(A_1 + A_2) + \frac{h^2}2 (A_1^2 + A_2^2 + A_1A_2+ A_2A_1) + \bigo(h^3).
    \end{aligned}
  \]
  Hence the normal splitting and Strang splitting have a local truncation error of $\bigo(h^2)$
  and $\bigo(h^3)$,
  \[
    \begin{aligned}
      \mathrm e^{h(A_1 + A_2)} - \mathrm e^{hA_2}\mathrm e^{hA_1} &= \bigo(h^2), \\
      \mathrm e^{h(A_1 + A_2)} - \mathrm e^{\frac h2A_1}\mathrm e^{hA_2}\mathrm e^{\frac h2A_1}
                                                                  &= \bigo(h^3). \\
    \end{aligned}
  \]

  \section{Pseudo inverse function}
  Providing that partial derivatives of $V$ are switchable, and let $M\equiv 1 / V'$, we have that
  \begin{equation}
    \label{eq:rho t circ V}
    (\partial_t\rho)\circ V = -M(M\partial_t V)'.
  \end{equation}
  Plugging \cref{eq:rho t circ V} into the power-law nonlinear diffusion system
  \[
    \partial_t \rho = \left(D_{\rho}\rho^{\gamma - 1}\rho' - \chi\rho c'\right)'
  \]
  gives us
  \[
    \partial_t V = -\frac{D_\rho}{\gamma}(M^\gamma)' + \chi c'\circ V.
  \]

  \section{FEM}
  Given the basis $\{\phi_l\}$ of the finite subspace, the weak formulation of problem
  \[
    \left\{
      \begin{aligned}
        \partial_t c &= D_c \partial_x^2 c + R_c(\rho, c), \\
        \partial_x c &= 0\quad\text{at}\;\{a,b\}
      \end{aligned}
    \right.
  \]
  reads
  \begin{equation}
    \label{eq:sys2 weak form}
    \int_a^b \partial_t c \phi_l\diff x = - D_c\int_a^b \partial_x c\phi_l'\diff x
                                          + \int_a^b R_c(\rho, c)\phi_l\diff x,
  \end{equation}
  as
  \[
    \int_a^b \partial^2_x c\phi_l \diff x = - \int_a^b \partial_x c\phi_l'\diff x.
  \]

  Let $c(x, t) = \sum_k c_k(t)\phi_k(x)$, then \cref{eq:sys2 weak form} writes
  \[
    \sum_k c'_k(t)\int_a^b\phi_k\phi_l\diff x = - D_c\sum_k c_k(t)\int_a^b\phi_k'\phi_l'\diff x
                                                + \int_a^b R_c(\rho, c)\phi_l\diff x.
  \]

  Given a mesh $\{x_i\},i = 0, 1, \ldots, M$, if we choose $\phi_i$ to be the hat function,
  then we have
  \[
    \begin{aligned}
      \int\phi_k\phi_l\diff x &=
      \begin{cases}
        \left|x_l - x_{l \pm 1}\right| / 2,     &k = l \in \{0, M\} \\
        \left|x_{l + 1} - x_{l - 1}\right| / 2, &k = l \not\in \{0, M\} \\
        \left|x_k - x_l\right| / 6,             &|k - l| = 1 \\
        0,                                      &|k - l| \geq 2
      \end{cases}, \\
      \int\phi_k'\phi_l'\diff x &=
      \begin{cases}
        1 / \left|x_l - x_{l \pm 1}\right|, &k = l \in \{0, M\} \\
        1 / \left|x_{l + 1} - x_l\right| + 1 / \left|x_l - x_{l - 1}\right|
                                            &k = l \not\in \{0, M\} \\
        -1 / \left|x_k - x_l\right|,        &|k - l| = 1 \\
        0,                                  &|k - l| \geq 2
      \end{cases}, \\
      \int\frac{\phi_k'}x\phi_l\diff x &=
      \begin{cases}
          \sum_{n = \pm}\frac{|x - x_n| - x_n\left|\ln x - \ln x_n\right|}{(x - x_n)^2}, &k = l\not\in\{0, M\} \\
          \frac{|x - x_n| - x\left|\ln x - \ln x_n\right|}{(x - x_n)^2}, & k = l \in\{0, M\} \\
          \frac{\sgn(k - l)}{(x_k - x_l)^2}\left(x_l - x_k - x_k\ln(x_l / x_k)\right), & |k - l| = 1 \\
          0, &\text{otherwise}
      \end{cases},
    \end{aligned}
  \]
  where $x_n$ denotes the corresponding neighbor point.

  \section{Implicit-explicit scheme}
  Suppose the PDE is $\partial_t y = f(y)$, where $f$ is an operator of $y$ (irrelevant to time
  variable $t$, not containing operations like $\partial_t$), then the scheme consists of two steps
  \[
    \left\{
      \begin{aligned}
        \frac{\tilde y - y}{\Delta t / 2} = f(\tilde y)\\
        \frac{T_{\Delta t}y - y}{\Delta t} = f(\tilde y),
      \end{aligned}
    \right.
  \]
  where $T_{\Delta t}y$ is the scheme approximation after time interval $\Delta t$.

  It is worth noting that the first step is implicit, so we apply the Newton's method to obtain
  the unknown intermediate state $\tilde y$. And if the operator $f$ contains terms that make
  it impossible to apply the Newton's method, we use $y$ instead.

  \section{Monotonicity preservation}
  For brevity we use the notation $\Delta V_{j + 1 / 2} = V_{j + 1} - V_j$,
  consider the forward Euler scheme of the form
  \begin{equation}
    \label{eq:Euler scheme}
    V_j(t + \Delta t) = V_j(t)
    + \Delta t\chi c'(V_j(t))
    - \frac{D_\rho\Delta t}{\gamma\Delta w}\left(\frac{\Delta w^\gamma}{(V_{j + 1}(t) - V_j(t))^\gamma}
      - \frac{\Delta w^\gamma}{(V_j(t) - V_{j - 1}(t))^\gamma}\right),
  \end{equation}
  then scheme is monotonicity preserving if $\exists\theta \in (0, 1)$, both CFL conditions
  \begin{subequations}
    \begin{align}
      \label{eq:CFL diffusion}
      \Delta t &< \frac{\theta}{2D_\rho \Delta w^{\gamma - 1}}
        \frac{\Delta V_{j + 1 / 2}(t)\Delta V_{j - 1 / 2}(t)}
          {\max_{k = j - 1, j}\left(\Delta V_{k + 1 / 2}(t)\right)^{-(\gamma - 1)}},
      \,\forall j \\
      \label{eq:CFL chemo}
      \Delta t &< \frac{1 - \theta}{\chi}\frac{\Delta V_{j + 1 / 2}(t)}{\left|c'(V_{j + 1}) - c'(V_j)\right|},
      \,\forall j
    \end{align}
  \end{subequations}
  are satistified.
  \begin{proof}
  We compute the difference
    \[
      \begin{aligned}
        \Delta V_{j + 1 / 2}(t + \Delta t) =& \Delta V_{j + 1 / 2}(t)
          + \Delta t\chi\left(c'(V_{j + 1}(t)) - c'(V_j(t))\right) \\
          &- \frac{\Delta tD_\rho}{\Delta w}\left(
              \frac{\left(\Delta w / \Delta V_{j + 3 / 2}(t)\right)^\gamma}{\gamma}
              - \frac{\left(\Delta w / \Delta V_{j + 1 / 2}(t)\right)^\gamma}{\gamma}
            \right) \\
          &+ \frac{\Delta tD_\rho}{\Delta w}\left(
              \frac{\left(\Delta w / \Delta V_{j + 1 / 2}(t)\right)^\gamma}{\gamma}
              - \frac{\left(\Delta w / \Delta V_{j - 1 / 2}(t)\right)^\gamma}{\gamma}
            \right).
      \end{aligned}
    \]
    By applying the mean value theorem to the function $f(x) = x^\gamma / \gamma$,
    we obtain
    \[
      \begin{aligned}
        \Delta V_{j + 1 / 2}(t + \Delta t) =& \Delta V_{j + 1 / 2}(t)
          + \Delta t\chi\left(c'(V_{j + 1}(t)) - c'(V_j(t))\right) \\
          &- \Delta tD_\rho\kappa_{j + 1}\left(
            \frac 1{\Delta V_{j + 3 / 2}(t)} - \frac 1{\Delta V_{j + 1 / 2}(t)}
            \right) \\
          &+ \Delta tD_\rho\kappa_j\left(
            \frac 1{\Delta V_{j + 1 / 2}(t)} - \frac 1{\Delta V_{j - 1 / 2}(t)}
            \right),
      \end{aligned}
    \]
    where $\kappa_j$ is between $\left(\Delta w / \Delta V_{j + 1 / 2}(t)\right)^{\gamma - 1}$
    and $\left(\Delta w / \Delta V_{j - 1 / 2}(t)\right)^{\gamma - 1}$.
    Then we define
    \[
      L_{j + 1 / 2} = \frac{c'(V_{j + 1}(t)) - c'(V'_j(t))}{V_{j + 1}(t) - V_j(t))},
    \]
    and rewrite
    \[
      \begin{aligned}
         &\Delta V_{j + 1 / 2}(t + \Delta t)\\
        =& \Delta V_{j + 1 / 2}(t)
        \left(
          1 + \Delta t\chi L_{j + 1 / 2}
          - \frac{\Delta t D_\rho\kappa_{j + 1}}{\Delta V_{j + 3 / 2}(t)\Delta V_{j + 1 / 2}(t)}
          - \frac{\Delta t D_\rho\kappa_j}{\Delta V_{j + 1 / 2}(t)\Delta V_{j - 1 / 2}(t)}
        \right)\\
         &+ \Delta t D_\rho\frac{\Delta t D_\rho(\kappa_j + \kappa_{j + 1})}{\Delta V_{j + 1 / 2}(t)}.
      \end{aligned}
    \]
    And by \cref{eq:CFL chemo} we have $1 - \Delta t\chi\left|L_{j + 1 / 2}\right| > \theta$,
    and by \cref{eq:CFL diffusion}
    \[
      \frac{\Delta t D_\rho \kappa_k}{\Delta V_{k + 1 / 2}(t)\Delta V_{k - 1 / 2}(t)}
      < \frac\theta 2,\quad k = j, j + 1.
    \]
    Hence $\Delta V_{j + 1 / 2}(t + \Delta t) > 0$ if $\Delta V_{j + 1 / 2}(t) > 0$, and scheme
    \ref{eq:Euler scheme} is monotonicity-preserving.
  \end{proof}

  For our splitting method, we avoid time step restrictions due to the diffusion terms by our
  implicit treatment and let $\theta = 0$.

  \section{Frechet derivative}
  For an operator $T: f\mapsto Tf$, by $\frender_fT$ (or $\frender_f(Tf)$ in case we omit the definition
  of $T$), we denote the Fr\'echet derivative
  of $T$ at $f$, and we can compute the derivatives of some useful operators,
  \[
    \begin{aligned}
      \frender_f(f')      &:g\mapsto g', \\
      \frender_f(h\circ f)&:g\mapsto g\cdot(h'\circ f), \\
      \frender_f(f\circ h)&:g\mapsto g\circ h. \\
    \end{aligned}
  \]
  And we have the chain rule
  \[
    \frender_f(TS) = \left(\frender_{Sf}T\right) \circ \left(\frender_fS\right),
  \]
  and the product rule
  \[
    \frender_f(T\cdot S) = Tf\cdot\frender_fS + Sf\cdot\frender_fT.
  \]

  Hence if we let
  \[
    T(V) = -\frac{D_\rho}{\gamma}\left[(V')^{-\gamma}\right]'
    + \chi c'\circ V,
  \]
  then its Fr\'echet derivative at $V$ writes
  \[
    \frender_V T: W\mapsto
    D_\rho\left[W'\cdot (V')^{-\gamma - 1}\right]' + \chi W\cdot(c''\circ V).
  \]
  Note that the function $c$ is not relevant to $V$ in the entity above,
  otherwise if we have an operator $T$ s.t.
  \[
    TV = c_V'\circ V = \int_{\tilde\Omega} f(V(\cdot) - V(w))\diff w,
  \]
  then
  \[
    \frender_VT:W \mapsto \int_{\tilde\Omega}\Big(W(\cdot) - W(w)\Big)f'(V(\cdot) - V(w))\diff w.
  \]

  And the Newton iteration writes
  \[
    (\frender_{V_k} T)(V_{k + 1} - V_k) = - TV_k.
  \]

  \subsection{Numerial expression}
  The numerial expression of $[W'\cdot(V')^{-\gamma - 1}]'$ writes
  \[
    (\Delta w)^{\gamma - 1}
    \left[
      \frac{1}{(V - V_-)^{\gamma + 1}}W_-
      -\left(\frac{1}{(V_+ - V)^{\gamma + 1}} + \frac{1}{(V - V_-)^{\gamma + 1}}\right)W
      +\frac{1}{(V_+ - V)^{\gamma + 1}}W_+
    \right],
  \]
  where $W_{\pm}$ represents the neighbor points of $W$.
  And for $[(V')^{-\gamma}]'$ it is
  \[
    (\Delta w)^{\gamma - 1}
    \left[
      \frac1{(V_+ - V)^{\gamma}} - \frac1{(V - V_-)^\gamma}
    \right].
  \]

  For $\int_{\tilde\Omega}(W(\cdot) - W(w))f'(V(\cdot) - V(w))\diff w$, it is
  \[
    \Delta w\left[\operatorname{diag}\left(f'(M_V)\vec 1\right) - f'(M_V)\right]W,
  \]
  where $M_V = V\vec 1^\top - \vec 1 V^\top$ denotes the cross matrix s.t. $M_{ij} = w_i - w_j$.
  Actually, the numerial expressions of $\int W(w)f(\cdot, w)\diff w$ and $\int W(\cdot)f(\cdot, w)\diff w$
  write $\Delta w FW$ and $\Delta w\operatorname{diag}(F\vec 1)W$ respectively,
  where $F_{ij} = F(x_i, w_j)$ denotes the cross matrix, and $w_j$ is the integral mesh while $x_i$ is
  the sample mesh.

  \section{Keller--Segel model with quadradic diffusion}
  We apply the splitting method to the Keller--Segel model with quadratic diffusion,
  which can be written as
  \begin{equation}
    \label{eq:ks quad}
    \left\{
      \begin{aligned}
        u_t &= (ru(u - \chi v)_r)_r, \\
        v_t &= v_{rr} + v_r / r - v + u.
      \end{aligned}
    \right.
  \end{equation}
  And the first equation in \cref{eq:ks quad} writes
  \[
    \partial_t\mathcal U = \frac{\mathcal U}{\mathcal U'}
                           (-1 / \mathcal U' + \chi v\circ\mathcal U)'
                         = \mathcal U\left[-\frac 12\left({\mathcal U'}^{-2}\right)'
                           + \chi v'\circ\mathcal U\right]
  \]
  by inverse cumulant formulation, where $\mathcal U'$ is the inverse cumulant
  function of $u$. And the second equation in FEM characterization writes
  \[
    \langle v_t, \phi\rangle
    = \langle (1 / r - 1)v_r, \phi\rangle + \langle u - v, \phi\rangle
  \]
  where $\phi$ is the basis function and $\langle\cdot, \cdot\rangle$ denotes
  the inner product.

  If we denote the right end of the first equation as operator $T$, then
  we can find its Fr\'echet derivative at $\mathcal U$ as
  \[
    \frender_{\mathcal U}T: \mathcal W\mapsto
    \left(\frac{\mathcal W'}{\mathcal U'^3}\right)'\mathcal U
    - \frac{\mathcal W}2\left(\frac1{\mathcal U'^2}\right)'
    +\chi\mathcal W\left[\mathcal U\cdot(v''\circ\mathcal U) + v'\circ\mathcal U\right]
  \]
\end{document}
